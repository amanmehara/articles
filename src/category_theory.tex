\documentclass{article}

\begin{document}

\title{Category Theory}
\author{Aman Mehara}
\maketitle

\section{Category}
A category is a collection of \textbf{objects} that are linked by \textbf{arrows}.

\section{Morphism}

\section{Universal Construction}

\section{Functor}
A \textbf{functor} is a map between categories.

\subsection{Definition}
Let C and D be categories. A \textbf{functor} F from C to D is a mapping that:

\begin{itemize}

\item 
  associates to each object \(X\) in C an object \(F(X)\) in D,

\item
  associates to each morphism \(f\colon X \to Y\) in C a morphism 
  \(F(f)\colon F(X) \to F(Y)\) in D such that the following two conditions hold:

  \begin{itemize} 

  \item 
    \(F(id_{x}) = id_{F(x)}\) for every object \(X\) in C.
    
  \item
    \(F(g \circ f) = F(g) \circ F(x)\) for all morphisms \(f\colon X \to Y\) and 
    \(g\colon Y \to Z\) in C.

  \end{itemize}

\end{itemize}

\section{Natural Transformations}
A \textbf{natural transformation} provides a way of transforming one functor 
into another while respecting the internal structure of the categories involved.

\subsection{Definition}
If F and G are functors between categories C and D, the a \textbf{natural 
transformation} \(\eta\) from F to G is a family of morpisms that satisfies two 
requirements.

\begin{itemize}

\item
  The natural transformation must associate to every object \(X\) in C a 
  morphism \(\eta_{X}\colon F(X) \to G(X)\) between objects of D. The morphism 
  \(\eta_{X}\) is called the component of \(\eta\) at \(X\).

\item
  Components must be such that for every morphism \(f\colon X \to Y\) in C we 
  have: \(\eta_{Y} \circ F(f) = G(f) \circ \eta_{X}\).

\end{itemize}

\section{Monad}

\end{document}
